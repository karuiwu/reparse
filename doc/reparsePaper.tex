%
% File acl2013.tex
%
% Contact  navigli@di.uniroma1.it
%%
%% Based on the style files for ACL-2012, which were, in turn,
%% based on the style files for ACL-2011, which were, in turn, 
%% based on the style files for ACL-2010, which were, in turn, 
%% based on the style files for ACL-IJCNLP-2009, which were, in turn,
%% based on the style files for EACL-2009 and IJCNLP-2008...

%% Based on the style files for EACL 2006 by 
%%e.agirre@ehu.es or Sergi.Balari@uab.es
%% and that of ACL 08 by Joakim Nivre and Noah Smith

\documentclass[11pt]{article}
\usepackage{acl2013}
\usepackage{times}
\usepackage{url}
\usepackage{latexsym}
\usepackage{verbatim}
%\setlength\titlebox{6.5cm}    % You can expand the title box if you
% really have to

\title{Title}

\author{Katherine Wu \\
  Affiliation / Address line 1 \\
  Affiliation / Address line 2 \\
  Affiliation / Address line 3 \\
  {\tt email@domain} \\\And
  Juneki Hong \\
  Affiliation / Address line 1 \\
  Affiliation / Address line 2 \\
  Affiliation / Address line 3 \\
  {\tt email@domain} \\\And
  Jason Eisner \\
  Affiliation / Address line 1 \\
  Affiliation / Address line 2 \\
  Affiliation / Address line 3 \\
  {\tt email@domain} \\
  }

\date{}



\begin{document}
\maketitle
\begin{abstract}

\end{abstract}





\section{Introduction}

\begin{itemize}
\item What dependency parsing is. Its a way to parse sentences to construct dependency trees.

\item Dependency parsers have issues when the "greedy" decision is not always optimal. In fact, parsing decisions may cause errors further down the line in things like garden path sentences. In order to parse these sentences well, dependency parsers may rely on a beam to search for an optimal parse. 

\item We introduce additional revision actions to be included with dependency parsers, that will increase our parsing accuracy and thus will allow parsers to run with smaller beam sizes for speed.


\end{itemize}



\section{Previous Work}

Deterministic approaches to dependency parsing have been \cite{Nivre03anefficient}.


\section{Arc Eager Dependency Parsing}



\section{Revision Actions}

We introduce a set of revision actions as additional choices an Arc Eager Dependency Parser could make. Revision operators will generally modify or remove existing arcs, and then backtrack by shifting words back on to the buffer to reparse the result. This is done with the expectation that parsing through a sentence a second time should be faster using the remaining arcs as chunked off partial parses. 

After the other revision operators are made we UNSHIFT. We move the partial parses back on to the buffer. 




\subsection{Generalized Arc Eager Dependency Parsing}
The Unshift revision action will move a set of words back on to the Parsing Buffer but it may also leave behind partial parses. 
%That is, arcs may be Unshifted back on to the buffer. 
Arc Eager Dependency Parsing by default assumes no previous arcs to exist on the buffer, and so additional parsing rules and constraints must be introduced in order to perform Generalized Arc Eager Dependency Parsing on tree inputs.

\begin{itemize}

\item If the top of the stack is the parent of the top of the buffer, then the RIGHT-ARC is \textbf{forced} to be made.

\item If the top of the buffer is the parent of the top of the stack, then the LEFT-ARC is \textbf{forced} to be made.

\item If the top of the buffer has a parent or if the top of the buffer has a left child on the stack, then a RIGHT-ARC \textbf{cannot} be made.

\item If the top of the stack has a parent or if the top of the stack has a right child on the buffer, then a LEFT-ARC \textbf{cannot} be made.

\item If the top of the stack has a parent or if the top of the buffer has a left child on the stack, then a REDUCE \textbf{must} be made.

\item If the top of the buffer has a left child on the stack then, a SHIFT \textbf{cannot} be made.

\end{itemize}





\section{Oracle Training}
In order to determine when the parser should use revision actions, an oracle was trained and used to determine when the parser had entered an error state.

\subsection{Parser State Features}
Features were extracted from the parser state to train the oracle. 

\begin{itemize}

\item POS N-Gram. The POS tags of the top N tokens on the stack and the top token on the buffer were taken.



\end{itemize}




\section{Experiments}

zPar is a dependency parser we used for our experiments. It had baseline results, and using our new revision actions, we were able to get improvements in performance even with a decrease in beam size.


\section{Analysis}


\section{Discussion}







\newpage






\begin{comment}


\section{examples of latex things:}

\newpage

\begin{itemize}
\item Left and right margins: 2.5 cm
\item Top margin: 2.5 cm
\item Bottom margin: 2.5 cm
\item Column width: 7.7 cm
\item Column height: 24.7 cm
\item Gap between columns: 0.6 cm
\end{itemize}

\noindent noindent ensures that there is no indent on a new paragraph.


\begin{table}[h]
\begin{center}
\begin{tabular}{|l|rl|}
\hline \bf Type of Text & \bf Font Size & \bf Style \\ \hline
paper title & 15 pt & bold \\
author names & 12 pt & bold \\
author affiliation & 12 pt & \\
the word ``Abstract'' & 12 pt & bold \\
section titles & 12 pt & bold \\
document text & 11 pt  &\\
captions & 11 pt & \\
abstract text & 10 pt & \\
bibliography & 10 pt & \\
footnotes & 9 pt & \\
\hline
\end{tabular}
\end{center}
\caption{\label{font-table} Font guide. }
\end{table}


{\bf Headings}: Type and label section and subsection headings in the
style shown on the present document.  Use numbered sections (Arabic
numerals) in order to facilitate cross references. Number subsections
with the section number and the subsection number separated by a dot,
in Arabic numerals. Do not number subsubsections.

{\bf Citations}: Citations within the text appear
in parentheses as~\cite{Gusfield:97} or, if the author's name appears in
the text itself, as Gusfield~\shortcite{Gusfield:97}. 
Append lowercase letters to the year in cases of ambiguity.  
Treat double authors as in~\cite{Aho:72}, but write as in~\cite{Chandra:81} when more than two authors are involved. Collapse multiple citations as in~\cite{Gusfield:97,Aho:72}. Also refrain from using full citations as sentence constituents. We suggest that instead of
\begin{quote}
  ``\cite{Gusfield:97} showed that ...''
\end{quote}
you use
\begin{quote}
``Gusfield \shortcite{Gusfield:97}   showed that ...''
\end{quote}

If you are using the provided \LaTeX{} and Bib\TeX{} style files, you
can use the command \verb|\newcite| to get ``author (year)'' citations.



\subsection{Graphics}

{\bf Illustrations}: Place figures, tables, and photographs in the
paper near where they are first discussed, rather than at the end, if
possible.  Wide illustrations may run across both columns.  Color
illustrations are discouraged, unless you have verified that  
they will be understandable when printed in black ink.

{\bf Captions}: Provide a caption for every illustration; number each one
sequentially in the form:  ``Figure 1. Caption of the Figure.'' ``Table 1.
Caption of the Table.''  Type the captions of the figures and 
tables below the body, using 11 point text.  
\end{comment}










\section*{Acknowledgments}

Do not number the acknowledgment section. Do not include this section
when submitting your paper for review.\cite{goldberg}

\bibliographystyle{acl}
% you bib file should really go here 
%\bibliography{acl2013}
\bibliography{reparse}


%\begin{thebibliography}{}

%\bibitem[\protect\citename{Goldberg and Nivre} 2012]{Goldberg:12}
%Yoav Goldberg and Joakim Nivre.
%\newblock 2012.
%\newblock {\em A Dynamic Oracle for Arc-Eager Dependency Parsing}

%\bibitem[\protect\citename{Aho and Ullman}1972]{Aho:72}
%Alfred~V. Aho and Jeffrey~D. Ullman.
%\newblock 1972.
%\newblock {\em The Theory of Parsing, Translation and Compiling}, volume~1.
%\newblock Prentice-{Hall}, Englewood Cliffs, NJ.

%\bibitem[\protect\citename{Gusfield}1997]{Gusfield:97}
%Dan Gusfield.
%\newblock 1997.
%\newblock {\em Algorithms on Strings, Trees and Sequences}.
%\newblock Cambridge University Press, Cambridge, UK.

%\end{thebibliography}

\end{document}
