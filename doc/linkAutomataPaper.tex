%
% File acl2013.tex
%
% Contact  navigli@di.uniroma1.it
%%
%% Based on the style files for ACL-2012, which were, in turn,
%% based on the style files for ACL-2011, which were, in turn, 
%% based on the style files for ACL-2010, which were, in turn, 
%% based on the style files for ACL-IJCNLP-2009, which were, in turn,
%% based on the style files for EACL-2009 and IJCNLP-2008...

%% Based on the style files for EACL 2006 by 
%%e.agirre@ehu.es or Sergi.Balari@uab.es
%% and that of ACL 08 by Joakim Nivre and Noah Smith

\documentclass[11pt]{article}
\usepackage{acl2013}
\usepackage{times}
\usepackage{url}
\usepackage{latexsym}
\usepackage{verbatim}
%\setlength\titlebox{6.5cm}    % You can expand the title box if you
% really have to

\title{Transition Based Link Grammar Parsing}



\author{Katherine Wu \\
  Affiliation / Address line 1 \\
  Affiliation / Address line 2 \\
  Affiliation / Address line 3 \\
  {\tt email@domain} \\\And
  Juneki Hong \\
  Affiliation / Address line 1 \\
  Affiliation / Address line 2 \\
  Affiliation / Address line 3 \\
  {\tt email@domain} \\\And
  Jason Eisner \\
  Affiliation / Address line 1 \\
  Affiliation / Address line 2 \\
  Affiliation / Address line 3 \\
  {\tt email@domain} \\
  }



\date{}



\begin{document}
\maketitle
\begin{abstract}


\end{abstract}





\section{Introduction}

%\item What dependency parsing is. Its a way to parse sentences to construct dependency trees.
There is a weakness to dependency parsing, because the action of taking additional children is in effect stateless. In other words, only a comparison of the top word in the stack is made with the top word in the buffer before a parsing decision is made, and this could lead to incorrect parses such as where Verbs have taken multiple subjects. 

Although Dependency parsers come with children features to approximate a solution to this (such as having a feature for the first or last child for example), this is not the same as knowing exactly which children have been attached. 

% having head automata be in particular states when certain children have and have not been accepted.

We propose having Link Grammar Automata features that will accept connecting children and emit costs to do so.

We modified Arc-Eager Dependency Parser Zpar (Zhang and Clark, 2011) to parse Link Grammars instead of Dependency Grammars, and implemented Link Grammar Automata features corresponding to each constituent of every sentence.





\section{Previous Work}

http://www.link.cs.cmu.edu/link/papers/index.html


\section{Link Grammars}
A link grammmar can form undirected edge connections between two constituents. 
We maintain projective parses, and a complete parse of a sentence will have no disconnected constituent.

\section{Link Automaton}
For each token of an input sentence, we will have an automaton that will change state as its constituent connects with other constituents in a parse. When a connection is formed, the automaton representing each constituents will accept each other and transition their states. Thus, a ``bad'' connection could be determined if the automatons had to take low probability or high cost transitions.

\section{Link Grammar Parsing}
We parse from left to right, using a stack and a buffer similar to Shift Reduce Parsing or Arc Eager Dependency Parsing (Although a bottom-up parse would also be possible, such as in CKY). In between the stack and buffer we can choose to do a CONNECT, SHIFT, or a REDUCE action. 


\section{Implementation}
We first prototyped our ideas by implementing the Link Automata features in the Dependency Parser. Zpar would use a perceptron to learn which Arc Eager action it should take for a given parsing state, and we added the Link Automatas for another available feature. (This in itself improved the performance of the parser by a small amount ...)

We then next changed Zpar to perform Link Grammar actions instead of Arc Eager actions. The perceptron training was no longer learning between taking the actions of ARC-LEFT, ARC-RIGHT, SHIFT, and REDUCE, but rather just the action CONNECT, SHIFT, and REDUCE.







\section{Experiments}

We used an annotated Link Grammar dataset, and we found a link grammar parser for our baseline. We then used our transition based Link Grammar Parser, and then we charted the performances.

Charts comparing performances on the length of the sentences. 

If the baseline uses training, we can have charts comparing accuracy vs size of the training data.

We can have recall and precision curves based on the annotated gold parses.


\section{Analysis}


\section{Discussion}



\section{Future Work}
We want to use our Link Grammar Parser and the Link Grammar Automaton as ``uncertainty'' features for revision actions. A low probability Link Automaton action, for example, would make the parser feel uncertain about its progress and it may choose to backtrack.





\newpage






\begin{comment}


\section{examples of latex things:}

\newpage

\begin{itemize}
\item Left and right margins: 2.5 cm
\item Top margin: 2.5 cm
\item Bottom margin: 2.5 cm
\item Column width: 7.7 cm
\item Column height: 24.7 cm
\item Gap between columns: 0.6 cm
\end{itemize}

\noindent noindent ensures that there is no indent on a new paragraph.


\begin{table}[h]
\begin{center}
\begin{tabular}{|l|rl|}
\hline \bf Type of Text & \bf Font Size & \bf Style \\ \hline
paper title & 15 pt & bold \\
author names & 12 pt & bold \\
author affiliation & 12 pt & \\
the word ``Abstract'' & 12 pt & bold \\
section titles & 12 pt & bold \\
document text & 11 pt  &\\
captions & 11 pt & \\
abstract text & 10 pt & \\
bibliography & 10 pt & \\
footnotes & 9 pt & \\
\hline
\end{tabular}
\end{center}
\caption{\label{font-table} Font guide. }
\end{table}


{\bf Headings}: Type and label section and subsection headings in the
style shown on the present document.  Use numbered sections (Arabic
numerals) in order to facilitate cross references. Number subsections
with the section number and the subsection number separated by a dot,
in Arabic numerals. Do not number subsubsections.

{\bf Citations}: Citations within the text appear
in parentheses as~\cite{Gusfield:97} or, if the author's name appears in
the text itself, as Gusfield~\shortcite{Gusfield:97}. 
Append lowercase letters to the year in cases of ambiguity.  
Treat double authors as in~\cite{Aho:72}, but write as in~\cite{Chandra:81} when more than two authors are involved. Collapse multiple citations as in~\cite{Gusfield:97,Aho:72}. Also refrain from using full citations as sentence constituents. We suggest that instead of
\begin{quote}
  ``\cite{Gusfield:97} showed that ...''
\end{quote}
you use
\begin{quote}
``Gusfield \shortcite{Gusfield:97}   showed that ...''
\end{quote}

If you are using the provided \LaTeX{} and Bib\TeX{} style files, you
can use the command \verb|\newcite| to get ``author (year)'' citations.



\subsection{Graphics}

{\bf Illustrations}: Place figures, tables, and photographs in the
paper near where they are first discussed, rather than at the end, if
possible.  Wide illustrations may run across both columns.  Color
illustrations are discouraged, unless you have verified that  
they will be understandable when printed in black ink.

{\bf Captions}: Provide a caption for every illustration; number each one
sequentially in the form:  ``Figure 1. Caption of the Figure.'' ``Table 1.
Caption of the Table.''  Type the captions of the figures and 
tables below the body, using 11 point text.  
\end{comment}











\bibliographystyle{acl}
% you bib file should really go here 
%\bibliography{acl2013}
\bibliography{linkAutomata}



\end{document}
